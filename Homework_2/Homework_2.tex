% !TEX program = pdflatex
% Statistical Physics Homework_2
\documentclass[12pt,a4paper]{article}
\usepackage[margin=1in]{geometry} 
\usepackage{amsmath,amsthm,amssymb,amsfonts,enumitem,fancyhdr,color,comment,graphicx,environ}
\pagestyle{fancy}
\setlength{\headheight}{65pt}
\newenvironment{problem}[2][Problem]{\begin{trivlist}
\item[\hskip \labelsep {\bfseries #1}\hskip \labelsep {\bfseries #2.}]}{\end{trivlist}}
\newenvironment{sol}
    {\emph{Solution:}
    }
    {
    \qed
    }
\specialcomment{com}{ \color{blue} \textbf{Comment:} }{\color{black}} %for instructor comments while grading
\NewEnviron{probscore}{\marginpar{ \color{blue} \tiny Problem Score: \BODY \color{black} }}
\usepackage[UTF8]{ctex}
\lhead{Name: 陈稼霖\\ StudentID: 45875852}
\rhead{PHYS1503 \\ Statistical Physics \\ Semester Fall 2019 \\ Assignment 2}
\begin{document}
\begin{problem}{2.2}
设一物质的物态方程具有以下的形式:
\[
p=f(v)T
\]
试证明其内能与体积无关。
\end{problem}
\begin{sol}
\begin{equation}
\left(\frac{\partial U}{\partial V}\right)_T=T\left(\frac{\partial p}{\partial T}\right)_V-p=Tf(v)-f(v)T=0
\end{equation}
故其内能与体积无关。
\end{sol}

\begin{problem}{2.3}
求证:(a)$\left(\frac{\partial S}{\partial p}\right)_H<0$;(b) $\left(\frac{\partial S}{\partial V}\right)_U>0$.
\end{problem}
\begin{sol}
\\(a)焓的全微分
\begin{equation}
dH=TdS+Vdp
\end{equation}
令$dH=0$,得
\begin{equation}
\left(\frac{\partial S}{\partial p}\right)_H=-\frac{V}{T}<0
\end{equation}
(b)内能的全微分
\begin{equation}
dU=TdS-pdV
\end{equation}
令$dU=0$,得
\begin{equation}
\left(\frac{\partial S}{\partial V}\right)_U=\frac{p}{T}>0
\end{equation}
\end{sol}

\begin{problem}{2.6}
试证明在相同的压强降落下,气体在准静态绝热膨胀中的温度降落大于在节流过程中的温度降落。\\
(提示:证明$\left(\frac{\partial T}{\partial p}\right)_S-\left(\frac{\partial T}{\partial p}\right)_H>0$)
\end{problem}
\begin{sol}
将熵视为温度$T$和压强$p$的函数
\begin{equation}
S=S(T,p)
\end{equation}
熵的全微分
\begin{equation}
dS=\left(\frac{\partial S}{\partial T}\right)_pdT+\left(\frac{\partial S}{\partial p}\right)_Tdp
\end{equation}
令$S=0$,得
\begin{equation}
\left(\frac{\partial T}{\partial p}\right)_S=-\frac{\left(\frac{\partial S}{\partial p}\right)_T}{\left(\frac{\partial S}{\partial T}\right)_p}
\end{equation}
根据定压热容量表达式
\begin{equation}
C_p=T\left(\frac{\partial S}{\partial T}\right)_p
\end{equation}
有
\begin{equation}
\left(\frac{\partial T}{\partial p}\right)_S=-\frac{T\left(\frac{\partial S}{\partial p}\right)_T}{C_p}
\end{equation}
焓的全微分
\begin{equation}
dH=\left(\frac{\partial H}{\partial T}\right)_pdT+\left(\frac{\partial H}{\partial p}\right)_Tdp
\end{equation}
令$dH=0$,得
\begin{equation}
\left(\frac{\partial T}{\partial p}\right)_H=-\frac{\left(\frac{\partial H}{\partial p}\right)_T}{\left(\frac{\partial H}{\partial T}\right)_p}
\end{equation}
根据定压热容量的定义
\begin{equation}
C_p=\left(\frac{\partial H}{\partial T}\right)_p
\end{equation}
和温度保持不变时焓随压强的变化率与物态方程的关系
\begin{equation}
\left(\frac{\partial H}{\partial p}\right)_T=V-\left(\frac{\partial V}{\partial T}\right)_p
\end{equation}
有
\begin{equation}
\left(\frac{\partial T}{\partial p}\right)_H=\frac{\left(\frac{\partial V}{\partial T}\right)_p-V}{C_p}
\end{equation}
从而
\begin{equation}
\left(\frac{\partial T}{\partial p}\right)_S-\left(\frac{\partial T}{\partial p}\right)_H=\frac{V}{C_p}>0
\end{equation}
故在相同的压强降落下,气体在准静态绝热膨胀中的温度降落大于在节流过程中的温度降落。
\end{sol}

\begin{problem}{2.7}
实验发现,一气体的压强$p$与比体积$v$的乘积及内能密度$u$都只是温度$T$的函数,即
\[
pV=f(T),\quad U=U(T)
\]
试根据热力学理论,讨论该气体的物态方程可能具有什么形式。
\end{problem}
\begin{sol}
设气体质量为$m$,则$V=mv$,从而
\begin{equation}
pV=p\cdot mv=mf(T)
\end{equation}
气体内能
\begin{equation}
U=Vu(T)
\end{equation}
一方面,气体内能在恒温下关于体积$V$的偏导
\begin{equation}
\left(\frac{\partial U}{\partial V}\right)_T=u(T)
\end{equation}
另一方面,温度保持不变时内能随体积的变化率与物态方程的关系
\begin{align}
\nonumber\left(\frac{\partial U}{\partial V}\right)_T=&T\left(\frac{\partial p}{\partial T}\right)_V-p=T\left(\frac{\partial\left(\frac{mf(T)}{V}\right)}{\partial T}\right)_V-\frac{mf(T)}{V}\\
=&\frac{mT}{V}\frac{df(T)}{dT}-\frac{mf(T)}{V}
\end{align}
故
\begin{equation}
u(T)=\frac{mT}{V}\frac{df(T)}{dT}-\frac{mf(T)}{V}
\end{equation}
内能密度在恒温下关于体积$V$的偏导
\begin{gather}
\left(\frac{\partial u(T)}{\partial V}\right)_T=-\frac{mT}{V^2}\frac{df(T)}{dT}+\frac{mf(T)}{V^2}=0\\
\Longrightarrow\frac{df(T)}{f(T)}=\frac{dT}{T}\\
\Longrightarrow\ln f(T)=\ln T+\ln C\\
\Longrightarrow pV=mCT
\end{gather}
\end{sol}

\begin{problem}{2.9}
试证明
\[
\left(\frac{\partial C_V}{\partial V}\right)_T=T\left(\frac{\partial^2p}{\partial T^2}\right)_V,\quad\left(\frac{\partial C_p}{\partial p}\right)_T=-T\left(\frac{\partial^2V}{\partial T^2}\right)_p
\]
并由此导出
\begin{gather*}
C_V=C_V^0+T\int_{V_0}^V\left(\frac{\partial^2p}{\partial T^2}\right)_VdV\\
C_p=C_p^0-T\int_{p_0}^p\left(\frac{\partial^2V}{\partial T^2}\right)_pdp
\end{gather*}
根据以上两式证明,理想气体的定容热容量和定压热容量只是温度$T$的函数。
\end{problem}
\begin{sol}
等容热容
\begin{equation}
C_V=T\left(\frac{\partial S}{\partial T}\right)_V
\end{equation}
$C_V$在恒温条件下关于体积$V$的偏导
\begin{equation}
\left(\frac{\partial C_V}{\partial V}\right)_T=T\left(\frac{\partial}{\partial V}\left(\frac{\partial S}{\partial T}\right)_V\right)_T=T\left(\frac{\partial}{\partial T}\left(\frac{\partial S}{\partial V}\right)_T\right)_V
\end{equation}
根据麦氏关系$\left(\frac{\partial S}{\partial V}\right)_T=\left(\frac{\partial p}{\partial T}\right)_V$
\begin{equation}
\left(\frac{\partial C_V}{\partial V}\right)_T=T\left(\frac{\partial^2p}{\partial T^2}\right)_V
\end{equation}
在恒温条件下方程两边同对体积$V$积分,得
\begin{equation}
C_V=C_V^0+T\int_{V_0}^V\left(\frac{\partial^2p}{\partial T^2}\right)_VdV
\end{equation}
等压热容
\begin{equation}
C_p=T\left(\frac{\partial S}{\partial T}\right)_p
\end{equation}
$C_p$在恒温条件下关于压强$p$的偏导
\begin{equation}
\left(\frac{\partial C_p}{\partial p}\right)_T=T\left(\frac{\partial}{\partial p}\left(\frac{\partial S}{\partial T}\right)_p\right)_T=T\left(\frac{\partial}{\partial T}\left(\frac{\partial S}{\partial p}\right)_T\right)_p
\end{equation}
根据麦氏关系$\left(\frac{\partial S}{\partial p}\right)_T=-\left(\frac{\partial V}{\partial T}\right)_p$
\begin{equation}
\left(\frac{\partial C_p}{\partial p}\right)_T=-T\left(\frac{\partial^2V}{\partial T^2}\right)_p
\end{equation}
在恒温条件下方程两边同对压强$p$积分,得
\begin{equation}
C_p=C_p^0-T\int_{p_0}^p\left(\frac{\partial^2V}{\partial T^2}\right)_pdp
\end{equation}
理想气体物态方程
\begin{equation}
pV=nRT
\end{equation}
从而
\begin{gather}
\left(\frac{\partial C_V}{\partial V}\right)_T=0\\
\left(\frac{\partial C_p}{\partial p}\right)_T=0
\end{gather}
故理想气体的定容热容量和定压热容量只是温度$T$的函数。
\end{sol}

\begin{problem}{2.13}
X射线衍射实验发现,橡皮带未被拉紧时具有无定型结构,当受张力而被拉伸时,具有晶体结构。这一事实表明橡皮带具有大的分子链。(a)试讨论橡皮带在等温过程中被拉伸时它的熵是增加还是减少;(b)试证明它的膨胀系数$\alpha=\frac{1}{L}\left(\frac{\partial L}{\partial T}\right)_F$是负的。
\end{problem}
\begin{sol}
\\(a)橡皮带拉紧后由无定型结构转变为晶体结构,有序程度增加,而熵是刻画系统无序程度的状态函数,因此橡皮带在等温过程中被拉伸时它的熵减少。\\
(b)橡皮带自由能的全微分(为与题设中的橡皮筋拉力$F$相区分,用字母$f$表示)
\begin{equation}
df=-SdT+FdL
\end{equation}
从而
\begin{gather}
\left(\frac{\partial f}{\partial T}\right)_L=-S\\
\left(\frac{\partial f}{\partial L}\right)_T=F
\end{gather}
考虑到偏导数的次序可以交换
\begin{gather}
\left(\frac{\partial}{\partial L}\left(\frac{\partial f}{\partial T}\right)_L\right)_T=-\left(\frac{\partial S}{\partial L}\right)_T=\left(\frac{\partial}{\partial T}\left(\frac{\partial f}{\partial L}\right)_T\right)_L=\left(\frac{\partial F}{\partial T}\right)_L
\end{gather}
根据(a)中结论
\begin{equation}
\left(\frac{\partial S}{\partial L}\right)_T<0
\end{equation}
故
\begin{equation}
\left(\frac{\partial F}{\partial T}\right)_L>0
\end{equation}
根据链式关系
\begin{equation}
\left(\frac{\partial L}{\partial T}\right)_F=\frac{-1}{\left(\frac{\partial T}{\partial F}\right)_L\left(\frac{\partial F}{\partial L}\right)_T}=\frac{-\left(\frac{\partial F}{\partial T}\right)_L}{\left(\frac{\partial F}{\partial L}\right)_T}
\end{equation}
根据生活常识,橡皮带受拉力伸长
\begin{equation}
\left(\frac{\partial F}{\partial L}\right)_T>0
\end{equation}
因此
\begin{equation}
\left(\frac{\partial L}{\partial T}\right)_F=\frac{-\left(\frac{\partial F}{\partial T}\right)_L}{\left(\frac{\partial F}{\partial L}\right)_T}<0
\end{equation}
故它的膨胀系数
\begin{equation}
\alpha=\frac{1}{L}(\frac{\partial L}{\partial T})_F<0
\end{equation}
\end{sol}

\begin{problem}{2.14}
假设太阳是黑体,根据下列数据求太阳表面的温度。\\
单位时间内透射到地球大气层外单位面积上的太阳辐射能量为$1.35\times10^3\text{J}\cdot\text{m}^{-2}\cdot\text{s}^{-1}$(该值称为太阳常数),太阳的半径为$6.955\times10^8$m,太阳与地球的平均距离为$1.495\times10^{11}$m。
\end{problem}
\begin{sol}
太阳的总辐射功率为
\begin{equation}
P=J_{u1}\cdot4\pi r^2=1.35\times10^3\text{J}\cdot\text{m}^{-2}\cdot\text{s}^{-1}\times4\pi(1.495\times10^{11}\text{m})^2=3.79\times10^{26}\text{W}
\end{equation}
太阳作为一个黑体,其表面的辐射通量密度
\begin{equation}
J_u=\frac{P}{4\pi R^2}=\frac{3.79\times10^{26}\text{W}}{4\pi(6.955\times10^8\text{m})^2}=6.24\times10^7\text{W}\cdot\text{m}^2
\end{equation}
根据斯特藩-玻尔兹曼定律,太阳表面的温度
\begin{equation}
T=\sqrt[4]{\frac{J_u}{\sigma}}=\sqrt[4]{\frac{6.24\times10^7\text{W}\cdot\text{m}^2}{5.669\times10^{-8}\text{W}\cdot\text{m}^{-2}\cdot\text{K}^{-4}}}=5.76\times10^3K
\end{equation}
\end{sol}

\begin{problem}{2.18}
试证明磁介质$C_H$与$C_M$之差等于
\[
C_H-C_M=\mu_0T\left(\frac{\partial H}{\partial T}\right)_M^2\left(\frac{\partial M}{\partial H}\right)_T
\]
\end{problem}
\begin{sol}
将磁介质内能视为温度$T$和总磁矩$m$的函数
\begin{equation}
U=U(T,m)
\end{equation}
磁介质内能的全微分
\begin{equation}
dU=\left(\frac{\partial U}{\partial T}\right)_mdT+\left(\frac{\partial U}{\partial m}\right)_Tdm
\end{equation}
又可写成
\begin{equation}
dU=TdS+\mu_0Hdm
\end{equation}
由熵的全微分表达式
\begin{equation}
dS=\left(\frac{\partial S}{\partial T}\right)_mdT+\left(\frac{\partial S}{\partial m}\right)_Tdm
\end{equation}
可得
\begin{equation}
dU=T\left(\frac{\partial S}{\partial T}\right)_mdT+\left[T\left(\frac{\partial S}{\partial m}\right)_T+\mu_0H\right]dm
\end{equation}
故
\begin{equation}
C_m=\left(\frac{\partial U}{\partial T}\right)_m=T\left(\frac{\partial S}{\partial T}\right)_m
\end{equation}
将磁介质的焓视为温度$T$和磁场强度$H$的函数(为与磁场强度$H$相区分,焓用字母$h$表示)
\begin{equation}
h=h(T,H)
\end{equation}
磁介质焓的全微分
\begin{equation}
dh=\left(\frac{\partial H}{\partial T}\right)_HdT+\left(\frac{\partial h}{\partial H}\right)_TdH
\end{equation}
又可写成
\begin{equation}
dh=TdS+\mu_0mdH
\end{equation}
由熵的全微分表达式
\begin{equation}
dS=\left(\frac{\partial S}{\partial T}\right)_HdT+\left(\frac{\partial S}{\partial H}\right)_TdH
\end{equation}
可得
\begin{equation}
dh=T\left(\frac{\partial S}{\partial T}\right)_HdT+\left[T\left(\frac{\partial S}{\partial H}\right)_T+\mu_0m\right]dH
\end{equation}
故
\begin{gather}
C_H=\left(\frac{\partial h}{\partial T}\right)_H=T\left(\frac{\partial S}{\partial T}\right)_H\\
\Longrightarrow C_H-C_m=T\left(\frac{\partial S}{\partial T}\right)_H-T\left(\frac{\partial S}{\partial T}\right)_m
\end{gather}
将磁介质的熵视为温度$T$和磁场强度$H$的函数
\begin{equation}
S=S(T,H)=S(T,m(T,H))
\end{equation}
磁介质熵在恒定磁场下关于温度$T$的偏导
\begin{equation}
\left(\frac{\partial S}{\partial T}\right)_H=\left(\frac{\partial S}{\partial T}\right)_m+\left(\frac{\partial S}{\partial m}\right)_T\left(\frac{\partial m}{\partial T}\right)_H
\end{equation}
从而
\begin{equation}
C_H-C_m=T\left(\frac{\partial S}{\partial m}\right)_T\left(\frac{\partial m}{\partial T}\right)_H=T\left(\frac{\partial S}{\partial m}\right)_T\left(\frac{\partial m}{\partial T}\right)_H
\end{equation}
利用麦氏关系$\left(\frac{\partial S}{\partial m}\right)_T=-\mu_0\left(\frac{\partial H}{\partial T}\right)_m$
\begin{equation}
C_H-C_m=-\mu_0T\left(\frac{\partial H}{\partial T}\right)_m\left(\frac{\partial m}{\partial T}\right)_H
\end{equation}
根据链式关系
\begin{equation}
\left(\frac{\partial m}{\partial T}\right)=\frac{-1}{\left(\frac{\partial T}{\partial H}\right)_m\left(\frac{\partial H}{\partial m}\right)_T}=-\left(\frac{\partial H}{\partial T}\right)_m\left(\frac{\partial m}{\partial H}\right)_T
\end{equation}
故
\begin{equation}
C_H-C_m=-\mu_0T\left(\frac{\partial H}{\partial T}\right)_m^2\left(\frac{\partial m}{\partial H}\right)_T
\end{equation}
\end{sol}

\begin{problem}{2.20}
已知超导体的磁感应强度$B=\mu_0(H+M)=0$,求证:\\
(a)$C_M$与$M$无关,只是$T$的函数,其中$C_M$是在磁化强度$M$保持不变的热容量。\\
(b)$U=\int C_MdT-\frac{\mu_0M^2}{2}+U_0$\\
(c)$S=\int\frac{C_M}{T}dT+S_0$
\end{problem}
\begin{sol}
\\(a)超导体的热容量
\begin{equation}
C_M=T\left(\frac{\partial S}{\partial T}\right)_M
\end{equation}
热容量在恒温下关于磁化强度$M$的偏导
\begin{equation}
\left(\frac{\partial C_M}{\partial M}\right)_T=\left(\frac{\partial}{\partial M}\left(T\frac{\partial S}{\partial T}\right)_M\right)_T=T\left(\frac{\partial}{\partial M}\left(\frac{\partial S}{\partial T}\right)_M\right)_T=T\left(\frac{\partial}{\partial T}\left(\frac{\partial S}{\partial M}\right)_T\right)_M
\end{equation}
根据磁介质的麦氏关系$\left(\frac{\partial S}{V\partial M}\right)_T=-\mu_0\left(\frac{\partial H}{\partial T}\right)_M$
\begin{equation}
\left(\frac{\partial C_M}{\partial M}\right)_T=-\mu_0TV\left(\frac{\partial}{\partial T}\left(\frac{\partial H}{\partial T}\right)_T\right)_M
\end{equation}
由
\begin{equation}
B=\mu_0(H+M)=0
\end{equation}
得
\begin{equation}
H=-M\Longrightarrow\left(\frac{\partial H}{\partial T}\right)_M=0
\end{equation}
从而
\begin{equation}
\left(\frac{\partial C_M}{\partial M}\right)_T=0
\end{equation}
故$C_M$与$M$无关,只是$T$的函数。\\
(b)由内能的全微分
\begin{equation}
dU=TdS+\mu_0Hdm=TdS-\mu_0VMdM
\end{equation}
得
\begin{equation}
\left(\frac{\partial U}{\partial M}\right)_T=-\mu_0VM
\end{equation}
内能的全微分又可写作
\begin{equation}
dU=\left(\frac{\partial U}{\partial T}\right)_MdT+\left(\frac{\partial U}{\partial M}\right)_MdM=C_MdT-\mu_0VM
\end{equation}
积分即得
\begin{equation}
U=\int C_MdT-\frac{\mu_0VM^2}{2}+U_0
\end{equation}
(c)超导体熵的全微分
\begin{equation}
dS=\left(\frac{\partial S}{\partial T}\right)_MdT+\left(\frac{\partial S}{\partial m}\right)_Tdm
\end{equation}
由超导体的热容量
\begin{equation}
C_M=T\left(\frac{\partial S}{\partial T}\right)_M
\end{equation}
得
\begin{equation}
\left(\frac{\partial S}{\partial T}\right)_M=\frac{C_M}{T}
\end{equation}
再结合超导体的麦氏关系$\left(\frac{\partial S}{\partial m}\right)_T=\mu_0\left(\frac{\partial H}{\partial T}\right)_M$,得
\begin{equation}
dS=\frac{C_M}{T}+\mu_0\left(\frac{\partial H}{\partial T}\right)_M
\end{equation}
其中
\begin{equation}
B=\mu_0(H+M)=0\Longrightarrow H=-M\Longrightarrow\left(\frac{\partial H}{\partial T}\right)_M=0
\end{equation}
故积分得
\begin{equation}
S=\int\frac{C_M}{T}dT+S_0
\end{equation}
\end{sol}
\end{document}