% !TEX program = pdflatex
% Statistical Physics Homework_3
\documentclass[12pt,a4paper]{article}
\usepackage[margin=1in]{geometry} 
\usepackage{amsmath,amsthm,amssymb,amsfonts,enumitem,fancyhdr,color,comment,graphicx,environ}
\pagestyle{fancy}
\setlength{\headheight}{65pt}
\newenvironment{problem}[2][Problem]{\begin{trivlist}
\item[\hskip \labelsep {\bfseries #1}\hskip \labelsep {\bfseries #2.}]}{\end{trivlist}}
\newenvironment{sol}
    {\emph{Solution:}
    }
    {
    \qed
    }
\specialcomment{com}{ \color{blue} \textbf{Comment:} }{\color{black}} %for instructor comments while grading
\NewEnviron{probscore}{\marginpar{ \color{blue} \tiny Problem Score: \BODY \color{black} }}
\usepackage[UTF8]{ctex}
\usepackage{tipa}
\lhead{Name: 陈稼霖\\ StudentID: 45875852}
\rhead{PHYS1503 \\ Statistical Physics \\ Semester Fall 2019 \\ Assignment 3}
\begin{document}
\begin{problem}{3.1}
证明下列平衡判据(假设$S>0$):
\begin{itemize}
\item[(a)] 在$S$、$V$不变的情形下,稳定平衡态的$U$最小。
\item[(b)] 在$S$、$p$不变的情形下,稳定平衡态的$H$最小。
\item[(c)] 在$H$、$p$不变的情形下,稳定平衡态的$S$最大。
\item[(d)] 在$F$、$V$不变的情形下,稳定平衡态的$T$最小。
\item[(e)] 在$G$、$p$不变的情形下,稳定平衡态的$T$最小。
\item[(f)] 在$U$、$S$不变的情形下,平衡态的$V$最小。
\item[(g)] 在$F$、$T$不变的情形下,平衡态的$V$最小。
\end{itemize}
\end{problem}
\begin{sol}
\begin{itemize}
\item[(a)] 根据热力学第二定律
\begin{equation}
\delta U\leq T\delta S+\delta W
\end{equation}
在$S$、$V$不变的情形下,$\delta S=0,\text{\textcrd}W=0$,从而
\begin{equation}
\delta U\leq0
\end{equation}
当$U$达到最小时,必有
\begin{equation}
\delta U=0
\end{equation}
系统无法发生任何宏观变化而处于稳定的平衡状态,故在$S$、$V$不变的情形下,稳定平衡态的$U$最小。
\item[(b)] 在$S$、$p$不变的情形下,$\delta S=0,\delta W=-p\delta V$,从而
\begin{gather}
\delta U\leq-p\delta V\\
\Longrightarrow\delta H=\delta U+p\delta V\leq 0
\end{gather}
当$H$达到最小时,必有
\begin{equation}
\delta H=0
\end{equation}
系统无法发生任何宏观变化而处于稳定的平衡状态,故在$S$、$p$不变的情形下,稳定平衡态的$H$最小。
\item[(c)] 由
\begin{equation}
H=U+pV
\end{equation}
得
\begin{equation}
\delta H=\delta U+V\delta p+p\delta V\leq T\delta S+\text{\textcrd}W+p\delta V+V\delta p
\end{equation}
在$H$、$p$不变的情形下,$\delta H=0,\delta p=0,\text{\textcrd}W=-pdV$,从而
\begin{equation}
T\delta S\geq0
\end{equation}
当$S$达到最小时,必有
\begin{equation}
\delta S=0
\end{equation}
系统无法发生任何宏观变化而处于稳定的平衡状态,故在$H$、$p$不变的情形下,稳定平衡态的$S$最大。
\item[(d)] 由
\begin{equation}
F=U-TS
\end{equation}
得
\begin{equation}
\delta F=\delta U-S\delta T-T\delta S\leq -S\delta T+\text{\textcrd}W
\end{equation}
在$F$、$V$不变的情形下,$\delta F=0,\text{\textcrd}W=0$,从而
\begin{equation}
S\delta T\leq 0
\end{equation}
当$T$达到最小时,必有
\begin{equation}
\delta T=0
\end{equation}
系统无法发生任何宏观变化而处于稳定的平衡状态,故在$F$、$V$不变的情形下,稳定平衡态的$T$最小。
\item[(e)] 由
\begin{equation}
G=U-TS+pV
\end{equation}
得
\begin{equation}
\delta G=\delta U-S\delta T-T\delta S+V\delta p+p\delta V\leq\text{\textcrd}W-S\delta T+V\delta p+p\delta V
\end{equation}
在$G$、$p$不变的情形下,$\delta G=0,\delta p=0,\text{\textcrd}W=-p\delta V$,从而
\begin{equation}
S\delta T\leq0
\end{equation}
当$T$达到最小时,必有
\begin{equation}
\delta T=0
\end{equation}
系统无法发生任何宏观变化而处于稳定的平衡状态,故在$G$、$p$不变的情形下,稳定平衡态的$T$最小。
\item[(f)] 根据热力学第二定律
\begin{equation}
\delta U\leq T\delta S+\text{\textcrd}W
\end{equation}
在$U$、$S$不变的情形下,$\delta U=0,\delta S=0$,从而
\begin{equation}
\text{\textcrd}W\geq0
\end{equation}
当$V$达到最小时,外界无法再压缩系统以对系统做功,从而
\begin{equation}
\text{\textcrd}W=0
\end{equation}
系统无法发生任何宏观变化而处于稳定的平衡状态,故在$U$、$S$不变的情形下,平衡态的$V$最小。
\item[(g)] 根据
\begin{equation}
F=U-TS
\end{equation}
得
\begin{equation}
\delta F=\delta U-S\delta T-T\delta S\leq -S\delta T+\text{\textcrd}W
\end{equation}
在$F$、$T$不变的情形下,$\delta F=0,\delta T=0$,从而
\begin{equation}
\text{\textcrd}W\geq0
\end{equation}
当$V$达到最小时,外界无法再压缩系统以对系统做功,从而
\begin{equation}
\text{\textcrd}W=0
\end{equation}
系统无法发生任何宏观变化而处于稳定的平衡状态,故在$F$、$T$不变的情形下,平衡态的$V$最小。
\end{itemize}
\end{sol}

\begin{problem}{3.5}
求证:
\[
\left(\frac{\partial U}{\partial n}\right)_{T,V}-\mu=-T\left(\frac{\partial\mu}{\partial T}\right)_{V,n}
\]
\end{problem}
\begin{sol}
开系的内能
\begin{equation}
U=G+TS+pV
\end{equation}
微分得
\begin{equation}
dU=TdS-pdV+\mu dn
\end{equation}
恒温恒容下关于$n$偏求导得
\begin{gather}
\left(\frac{\partial U}{\partial n}\right)_{T,V}=T\left(\frac{\partial S}{\partial n}\right)_{T,V}+\mu\\
\label{3.5}\Longrightarrow\left(\frac{\partial U}{\partial n}\right)_{T,V}-\mu=T\left(\frac{\partial S}{\partial n}\right)_{T,V}
\end{gather}
开系的自由能的微分
\begin{equation}
dF=-SdT-pdV+\mu dn
\end{equation}
恒温恒容下关于$n$求偏导得
\begin{equation}
\left(\frac{\partial F}{\partial n}\right)_{T,V}=\mu
\end{equation}
$\mu$在恒容闭系条件下关于$T$求偏导
\begin{equation}
\left(\frac{\partial\mu}{\partial T}\right)_{V,n}=\left(\frac{\partial}{\partial T}\left(\frac{\partial F}{\partial n}\right)_{T,V}\right)_{V,n}=\left(\frac{\partial}{\partial n}\left(\frac{\partial F}{\partial T}\right)_{V,n}\right)_{T,V}=-\left(\frac{\partial S}{\partial n}\right)_{T,V}
\end{equation}
将上式回代式(\ref{3.5})即得
\begin{equation}
\left(\frac{\partial U}{\partial n}\right)_{T,V}-\mu=-T\left(\frac{\partial\mu}{\partial T}\right)_{V,n}
\end{equation}
\end{sol}

\begin{problem}{3.7}
试证明在相变中物质的摩尔内能的变化为
\[
\Delta U_m=L(1-\frac{p}{T}\frac{dT}{dp})
\]
如果一相是气相,可看做理想气体,另一相是凝聚相,试将公式化简。
\end{problem}
\begin{sol}
物质内能
\begin{equation}
U_m=H_m-pV
\end{equation}
当发生相变时,温度和压强不变,
\begin{equation}
\Delta U_m=\Delta H_m-p\Delta V_m=L-p\Delta V_m
\end{equation}
将克拉珀龙方程
\begin{equation}
\frac{dT}{dp}=\frac{T\Delta V_m}{L}
\end{equation}
代入即得
\begin{equation}
\Delta U_m=L(1-\frac{p}{T}\frac{dT}{dp})
\end{equation}
当一相是气相,可看做理想气体,另一相是凝聚相,则克拉珀龙方程可简化为
\begin{equation}
\frac{dp}{dT}=\frac{Lp}{RT^2}
\end{equation}
代入可得
\begin{equation}
\Delta U_m=L(1-\frac{RT}{L})
\end{equation}
\end{sol}

\begin{problem}{3.9}
以$c_{\alpha}^{\beta}$表示在维持$\beta$相和$\alpha$两相平衡的条件下$1$mol$\beta$相物质升高$1$K所吸收的热量,称为$\beta$相的两相平衡摩尔热容量。试证明
\[
c_{\alpha}^{\beta}=c_p^{\alpha}-\frac{L}{V_m^{\beta}-V_m^{\alpha}}\left(\frac{\partial V_m^{\beta}}{\partial T}\right)
\]
如果$\beta$相是蒸汽,可看作理想气体,$\alpha$是凝聚相,上式可化简为
\[
c_{\alpha}^{\beta}=c_p^{\beta}-\frac{L}{T}
\]
并说明为什么饱和蒸汽的热容量有可能是负的。
\end{problem}
\begin{sol}
热容量定义
\begin{equation}
c_{\beta}^{\alpha}=\lim_{dT\rightarrow0}\frac{\text{\textcrd}Q}{dT}
\end{equation}
对可逆过程
\begin{equation}
dS_m^{\beta}=\frac{\text{\textcrd}Q}{T}
\end{equation}
从而
\begin{equation}
c_{\beta}^{\alpha}=\lim_{dT\rightarrow}T\frac{dS_m^{\beta}}{dT}=T\frac{dS_m^{\beta}}{dT}=T\left(\frac{\partial S_m^{\beta}}{\partial T}\right)_p+T\left(\frac{\partial S_m^{\beta}}{\partial p}\right)_T\frac{dp}{dT}
\end{equation}
其中
\begin{gather}
T\left(\frac{\partial S_m^{\beta}}{\partial T}\right)_p=C_p^{\beta}\\
\left(\frac{\partial S_m^{\beta}}{\partial p}\right)_T=-\left(\frac{\partial V_m^{\beta}}{\partial T}\right)_p\\
\frac{dp}{dT}=\frac{L}{T(V_m^{\beta}-V_m^{\alpha})}
\end{gather}
代入即得
\begin{equation}
c_{\alpha}^{\beta}=C_p^{\beta}-\frac{L}{V_m^{\beta}-V_m^{\alpha}}\left(\frac{\partial V_m^{\beta}}{\partial T}\right)_p
\end{equation}
若$\beta$相可视为理想气体,则
\begin{equation}
p^{\beta}V_m^{\beta}=RT\Longrightarrow\left(\frac{dV_m^{\beta}}{dT}\right)_p=\frac{R}{p_m^{\beta}}
\end{equation}
$\alpha$是凝聚相,则$V_m^{\alpha}=0$可忽略,故
\begin{equation}
c_{\alpha}^{\beta}=c_p^{\beta}-\frac{LR}{V_m^{\beta}p_m^{\beta}}=c_p^{\beta}-\frac{L}{T}
\end{equation}
当$c_p^{\beta}<\frac{L}{T}$时,$c_{\alpha}^{\beta}$为负。
\end{sol}

\begin{problem}{3.11}
根据式(3.4.7) $\frac{1}{p}\frac{dp}{dT}=\frac{L}{RT^2}$ ,利用上题的结果 $\frac{dL}{dT}=c_p^{\alpha}-c_p^{\beta}$ 计及潜热 $L$ 是温度的函数,但假设温度的变化范围不大,定压热容量可以看作常数,证明蒸汽压方程可以表为
\[
\ln p=A-\frac{B}{T}+C\ln T
\]
\end{problem}
\begin{sol}
由
\begin{equation}
\frac{dL}{dT}=c_p^{\alpha}-c_p^{\beta}
\end{equation}
且定压热容量可视为常数的假设,积分得
\begin{equation}
L=L_0+(c_p^{\alpha}-c_p^{\beta})T
\end{equation}
代入
\begin{equation}
\frac{1}{p}\frac{dp}{dT}=\frac{L}{RT^2}
\end{equation}
得
\begin{equation}
\frac{dp}{p}=\frac{L_0}{RT^2}dT+\frac{c_p^{\alpha}-c_p^{\beta}}{RT}dT
\end{equation}
两边积分得
\begin{equation}
\ln p=A-\frac{L_0}{RT}+\frac{c_p^{\alpha}-c_p^{\beta}}{R}\ln T
\end{equation}
即
\begin{equation}
\ln p=A-\frac{B}{T}+C\ln T
\end{equation}
\end{sol}

\begin{problem}{3.16}
证明爱伦菲斯特公式
\begin{gather*}
\frac{dp}{dT}=\frac{\alpha^{(2)}-\alpha^{(1)}}{\kappa^{(2)}-\kappa^{(1)}}\\
\frac{dp}{dT}=\frac{c_p^{(2)}-c_p^{(1)}}{Tv(\alpha^{(2)}-\alpha^{(1)})}
\end{gather*}
\end{problem}
\begin{sol}
二级相变在临近的相变点$(T,p)$和$(T+dT,p+dp)$两相的比熵和比体积变化相等
\begin{gather}
dS_m^{(1)}=dS_m^{(2)}\\
dV_m^{(1)}=dV_m^{(2)}
\end{gather}
将比熵视为温度和压强的函数
\begin{equation}
S_m=S_m(T,p)
\end{equation}
微分得
\begin{align}
\nonumber ds=&\left(\frac{\partial v}{\partial T}\right)_pdT+\left(\frac{\partial s}{\partial p}\right)_Tdp\\
\nonumber =&\frac{C_p}{T}dT-\left(\frac{\partial V_m}{\partial T}\right)_pdp\\
=&\frac{C_p}{T}dT-V_m\alpha dp
\end{align}
从而
\begin{gather}
\frac{C_p^{(1)}}{T}dT-V_m^{(1)}\alpha^{(1)}dp=\frac{C_p^{(2)}}{T}dT-V_m^{(2)}\alpha^{(2)}dp\\
\Longrightarrow\frac{dp}{dT}=\frac{C_p^{(2)}-C_p^{(1)}}{TV_m(\alpha^{(2)}-\alpha^{(1)})}
\end{gather}
将比体积也视为温度和压强的函数
\begin{equation}
V_m=V_m(T,p)
\end{equation}
微分得
\begin{align}
\nonumber dV_m=&\left(\frac{\partial V_m}{\partial T}\right)_pdT+\left(\frac{\partial V_m}{\partial p}\right)_Tdp\\
=&V_m\alpha dT-V_m\kappa dp
\end{align}
从而
\begin{gather}
V_m^{(1)}\alpha^{(1)} dT-V_m^{(1)}\kappa^{(1)}dp=V_m^{(2)}\alpha^{(2)}dT-V_m^{(2)}\kappa^{(2)}dp
\end{gather}
由于二级相变中两相的比体积相等,$V_m^{(1)}=V_m^{(2)}$,故
\begin{gather}
\alpha^{(1)} dT-\kappa^{(1)}dp=\alpha^{(2)}dT-\kappa^{(2)}dp\\
\Longrightarrow\frac{dp}{dT}=\frac{\alpha^{(2)}-\alpha^{(1)}}{\kappa^{(2)}-\kappa^{(1)}}
\end{gather}
\end{sol}

\begin{problem}{4.1}
若将$U$看作独立变数$T,V,n_1,\cdots,n_k$的函数,试证明:
\begin{itemize}
\item[(a)] $U=\sum_in_i\frac{\partial U}{\partial n_i}+V\frac{\partial U}{\partial V}$
\item[(b)] $u_i=\frac{\partial U}{\partial n_i}+v_i\frac{\partial U}{\partial V}$
\end{itemize}
\end{problem}
\begin{sol}
\begin{itemize}
\item[(a)] $U$为$V,n_1,\cdots,n_k$的一次齐函数
\begin{equation}
U(T,\lambda V,\lambda n_1,\cdots,\lambda n_k)=\lambda U(T,V,n_1,\cdots,n_k)
\end{equation}
根据欧拉定理,上式两边同关于$\lambda$求导即得
\begin{equation}
V\frac{\partial U}{\partial V}+\sum_{i=1}^kn_i\frac{\partial U}{\partial n_i}=U
\end{equation}
\item[(b)] (a)中结论式关于$n_i$求偏导得
\begin{equation}
\label{4.1}u_i=\frac{\partial U}{\partial n_i}=\frac{\partial U}{\partial n_i}+\frac{\partial V}{\partial n_i}\frac{\partial U}{\partial V}
\end{equation}
总体积
\begin{equation}
V=\sum_{i=1}^kn_iv,i
\end{equation}
在定温定压其他物质摩尔数不变的条件下,关于$n_i$求偏导得
\begin{equation}
v_i=\frac{\partial V}{\partial n_i}
\end{equation}
代入式(\ref{4.1})中即得
\begin{equation}
u_i=\frac{\partial U}{\partial n_i}+v_i\frac{\partial U}{\partial V}
\end{equation}
\end{itemize}
\end{sol}

\begin{problem}{4.3}
二元理想溶液具有下列形式的化学势:
\begin{gather*}
\mu_1=g_1(T,p)+RT\ln x_1\\
\mu_2=g_2(T,p)+RT\ln x_2
\end{gather*}
其中$g_i(T,p)$为纯$i$组元的化学势,$x_i$是溶液中$i$组元的摩尔分数。当物质的量分别为$n_1$、$n_2$的两种纯液体在等温等压下合成理想溶液时,试证明混合前后
\begin{itemize}
\item[(a)] 吉布斯函数的变化为
\[
\Delta G=RT(n_1\ln x_1+n_2\ln x_2)
\]
\item[(b)] 体积不变$\Delta V=0$
\item[(c)] 熵变$\Delta S=-R(n_1\ln x_1+n_2\ln x_2)$
\item[(d)] 焓变$\Delta H=0$,因而没有混合热。
\item[(e)] 内能变化为何?
\end{itemize}
\end{problem}
\begin{sol}
\begin{itemize}
\item[(a)] 混合前,$1$组元和$2$组元的吉布斯函数分别为
\begin{gather}
G_1=n_1g_1(T,p)\\
G_2=n_2g_2(T,p)
\end{gather}
混合后,二元理想溶液的吉布斯函数
\begin{equation}
G=n_1\mu_1+n_2\mu_2=n_1[g_1(T,p)+RT\ln x_1]+n_2[g_2(T,p)+RT\ln x_2]
\end{equation}
混合前后吉布斯函数的变化为
\begin{equation}
\Delta G=G-G_1-G_2=RT(n_1\ln x_1+n_2\ln x_2)
\end{equation}
\item[(b)] 混合前,$1$组元和$2$组元的体积分别为
\begin{gather}
V_1=\frac{\partial G_1}{\partial p}=n_1\frac{\partial g_1(T,p)}{\partial p}\\
V_2=\frac{\partial G_2}{\partial p}=n_2\frac{\partial g_2(T,p)}{\partial p}
\end{gather}
混合后,二元理想溶液的体积
\begin{equation}
V=\frac{\partial G}{\partial p}=n_1\frac{\partial g_1(T,p)}{\partial p}+n_2\frac{\partial g_2(T,p)}{\partial p}
\end{equation}
混合前后,体积不变
\begin{equation}
\Delta V=V-V_1-V_2=0
\end{equation}
\item[(c)] 混合前,$1$组元和$2$组元的熵分别为
\begin{gather}
S_1=-\frac{\partial G}{\partial T}=-n_1\frac{\partial g_1(T,p)}{\partial T}\\
S_2=-\frac{\partial G}{\partial T}=-n_2\frac{\partial g_2(T,p)}{\partial T}
\end{gather}
混合后,二元理想溶液的熵
\begin{equation}
S=-\frac{\partial G}{\partial T}=-n_1\left[\frac{\partial g_1(T,p)}{\partial T}+R\ln x_1\right]-n_2\left[\frac{\partial g_2(T,p)}{\partial T}+R\ln x_2\right]
\end{equation}
混合前后,熵变
\begin{equation}
\Delta S=S-S_1-S_2=-R(n_1\ln x_1+n_2\ln x_2)
\end{equation}
\item[(d)] 混合前,$1$组元和$2$组元的焓分别为
\begin{gather}
H_1=G_1+S_1T=n_1g_1(T,p)-n_1\frac{\partial g_1(T,p)}{\partial T}\\
H_2=G_1+S_2T=n_2g_2(T,p)-n_2\frac{\partial g_2(T,p)}{\partial T}
\end{gather}
混合后,二元理想溶液的焓
\begin{equation}
H=G+S=n_1\left[g_1(T,p)-\frac{\partial g_1(T,p)}{\partial T}\right]+n_2\left[g_2(T,p)-\frac{\partial g_2(T,p)}{\partial T}\right]
\end{equation}
混合前后,焓不变
\begin{equation}
\Delta H=H-H_1-H_2=0
\end{equation}
因而没有混合热。
\item[(e)] 混合前,$1$组元和$2$组元的内能分别为
\begin{gather}
U_1=H_1-pV_1=n_1g_1(T,p)-n_1\frac{\partial g_1(T,p)}{\partial T}-pn_1\frac{\partial g_1(T,p)}{\partial p}\\
U_2=H_1-pV_2=n_2g_2(T,p)-n_2\frac{\partial g_2(T,p)}{\partial T}-pn_2\frac{\partial g_2(T,p)}{\partial p}
\end{gather}
混合后,二元理想溶液的内能为
\begin{align}
\nonumber U=&H-pV=n_1\left[g_1(T,p)-\frac{\partial g_1(T,p)}{\partial T}\right]+n_2\left[g_2(T,p)-\frac{\partial g_2(T,p)}{\partial T}\right]\\
&-p\left[n_1\frac{\partial g_1(T,p)}{\partial p}+n_2\frac{\partial g_2(T,p)}{\partial p}\right]
\end{align}
混合前后,内能不变
\begin{equation}
\Delta U=U-U_1-U_2=0
\end{equation}
\end{itemize}
\end{sol}

\begin{problem}{4.6}
如图4.6所示,开口玻璃管底端有半透膜将管中糖的水溶液与容器内的水隔开。半透膜只让水透过,不让糖透过。实验发现,糖水溶液的液面比例的液面比容器内水的液面上升一个高度$h$,表明糖水溶液的压强$p$与水的压强$p_0$之差$p-p_0=\rho gh$。这一压强差称为渗透压。试证明,糖水与水达到平衡时有
\[
g_1(T,p)+RT\ln(1-x)=g_1(T,p_0)
\]
其中$g_1$为纯水的摩尔吉布斯函数,$x$是糖水中糖的摩尔分数,$x=\frac{n_2}{n_1+n_2}\approx\frac{n_2}{n_1}\ll1$。试据此证明
\[
p-p_0=\frac{n_2RT}{V}
\]
其中$V$是糖水溶液的体积。
\end{problem}
\begin{sol}
将糖水视为混合理想气体,其中水的化学势为
\begin{equation}
\mu_1=RT[\varphi_i(T)+\ln p_1]
\end{equation}
关于水的摩尔分数$x^L=1-x$求导得
\begin{equation}
\left(\frac{\partial\mu_1}{\partial x^L}\right)_{T,p}=\left(\frac{\partial\mu_1}{\partial p_i}\right)_{T,p}\left(\frac{\partial p_i}{\partial x^L}\right)_{T,p}=\frac{RT}{p_1}\left(\frac{\partial p_1}{\partial x^L}\right)_{T,p}=RT\frac{\partial}{\partial x^L}\ln p_1=RT\frac{\partial}{\partial x^L}\ln xp=\frac{RT}{x^L}
\end{equation}
积分得糖水中水的化学势
\begin{equation}
\mu_1=g_1(T,p)+RT\ln x^L=g_1(T,p)+RT\ln(1-x)
\end{equation}
令$x^L\rightarrow1$,则$\mu_1=g_1(T,p)$,故$g_1(T,p)$为纯水的化学势。水作为能透过半透膜的组员,其在膜两边的化学势相等,即
\begin{equation}
g_1(T,p)+RT\ln(1-x)=g_1(T,p_0)
\end{equation}
对$g_1(T,p)$做一阶泰勒展开
\begin{equation}
g_1(T,p)-g_1(T,p_0)=\left(\frac{\partial g_1}{\partial p}\right)_T(p-p_0)=V_m(p-p_0)
\end{equation}
其中$V_m$为纯水的摩尔体积。
\begin{equation}
\Longrightarrow p-p_0=\frac{g_1(T,p)-g_1(T,p_0)}{V_m}=-\frac{RT\ln(1-x)}{V_m}
\end{equation}
由于$x=\frac{n_2}{n_1+n_2}\approx\frac{n_2}{n_1}\ll1$,有近似$\ln(1-x)=-\frac{n_2}{n_1}$,故
\begin{equation}
p-p_0=\frac{n_2RT}{n_1V_m}=\frac{n_2RT}{V}
\end{equation}
\end{sol}

\begin{problem}{4.8}
绝热容器中有隔板隔开,两边分别装有$n_1$mol和$n_2$mol的理想气体,温度同为$T$,压强分别为$p_1$和$p_2$。今将隔板抽去,
\begin{itemize}
\item[(a)] 试求气体混合后的压强;
\item[(b)] 如果两种气体是不同的,计算混合后的熵增;
\item[(c)] 如果两种气体是相同的,计算混合后的熵增。
\end{itemize}
\end{problem}
\begin{sol}
\begin{itemize}
\item[(a)] 隔板两边的体积分别为
\begin{gather}
V_1=\frac{n_1RT}{p_1}\\
V_2=\frac{n_2RT}{p_2}
\end{gather}
隔板抽去后的总体积
\begin{equation}
V=V_1+V_2=RT(\frac{n_1}{p_1}+\frac{n_2}{p_2})
\end{equation}
气体混合后的压强为
\begin{equation}
p=\frac{(n_1+n_2)RT}{V}=\frac{(n_1+n_2)p_1p_2}{(n_1p_2+n_2p_1)}
\end{equation}
\item[(b)] 若两种气体不同,混合前两种气体的熵分别为
\begin{gather}
S_1=n_1\left[\int c_{p1}\frac{dT}{T}-R\ln p_1+s_{10}\right]=n_1(c_{p1}\ln T-R\ln p_1+s_{10})\\
S_2=n_2\left[\int c_{p2}\frac{dT}{T}-R\ln p_2+s_{20}\right]=n_2(c_{p2}\ln T-R\ln p_2+s_{20})
\end{gather}
混合后气体的熵为
\begin{equation}
S=n_1(c_{p1}\ln T-R\ln\frac{n_1}{n_1+n_2}p+s_{10})+n_2(c_{p2}\ln T-R\ln\frac{n_2}{n_1+n_2}p+s_{20})
\end{equation}
混合后的熵增加值为
\begin{align}
\nonumber\Delta S=&S-S_1-S_2=-n_1R\ln\left(\frac{n_1}{n_1+n_2}\frac{p}{p_1}\right)-n_2R\ln\left(\frac{n_2}{n_1+n_2}\frac{p}{p_2}\right)\\
=&R\left[n_1\ln\left(\frac{n_1p_2+n_2p_1}{n_1p_2}\right)+n_2\ln\left(\frac{n_1p_2+n_2p_1}{n_2p_1}\right)\right]
\end{align}
\item[(c)] 若两种气体相同,混合前两种气体的熵分别为
\begin{gather}
S_1=n_1(c_p\ln T-R\ln p_1+s_0)\\
S_2=n_2(c_p\ln T-R\ln p_2+s_0)
\end{gather}
混合后气体的熵为
\begin{equation}
S=(n_1+n_2)(c_p\ln T-R\ln p+s_0)
\end{equation}
混合后的熵增加值为
\begin{align}
\nonumber\Delta S=&S-S_1-S_2=-R(n_1\ln\frac{p}{p_1}+n_2\ln\frac{p}{p_2})\\
=&R\left[n_1\ln\left(\frac{n_1p_2+n_2p_1}{(n_1+n_2)p_2}\right)+n_2\ln\left(\frac{n_1p_2+n_2p_1}{(n_1+n_2)p_1}\right)\right]
\end{align}
\end{itemize}
\end{sol}

\begin{problem}{4.9}
试证明,在NH$_3$分解为N$_2$和H$_2$的反应中
\[
\frac{1}{2}\text{N}_2+\frac{3}{2}\text{H}_2-\text{NH}_3=0
\]
平衡常量可表为
\[
K_p=\frac{\sqrt{27}}{4}\times\frac{\varepsilon^2}{1-\varepsilon^2}p
\]
如果将反应方程式写作
\[
\text{N}_2+3\text{H}_2-2\text{NH}_3=0
\]
平衡量为何?
\end{problem}
\begin{sol}
对于第一个方程,根据平衡常量的定义
\begin{equation}
\label{4.9}K_p=p^{\nu}\prod_ix_i^{\nu_i}=\frac{x_1^{\frac{1}{2}}x_2^{\frac{3}{2}}}{x_3}p
\end{equation}
设原有$n_0$mol的NH$_3$,分解度为$\varepsilon$,则分解后各物质量
\begin{gather}
n_1=\frac{1}{2}\varepsilon n_0\\
n_2=\frac{3}{2}\varepsilon n_0\\
n_3=(1-\varepsilon)n_0
\end{gather}
分解后总物质量
\begin{equation}
n=n_1+n_2+n_3=(1+\varepsilon)n_0
\end{equation}
分解后各物质摩尔分数
\begin{gather}
x_1=\frac{n_1}{n}=\frac{\varepsilon}{2(1+\varepsilon)}\\
x_2=\frac{n_2}{n}=\frac{3\varepsilon}{2(1+\varepsilon)}\\
x_3=\frac{n_3}{n}=\frac{1-\varepsilon}{1+\varepsilon}\\
\end{gather}
代入式(\ref{4.9})得
\begin{equation}
K_p=\frac{3\sqrt{3}}{4}\frac{\varepsilon^2}{1-\varepsilon^2}p
\end{equation}
对于第二个方程,根据平衡常量的定义
\begin{equation}
K_p=p^{\nu}\prod_ix_i^{\nu_i}=\frac{x_1x_2^{3}}{x_3^2}p^2=\frac{27}{16}\frac{\varepsilon^4}{(1-\varepsilon^2)^2}p^2
\end{equation}
\end{sol}

\begin{problem}{4.10}
$n_0\nu_1$mol的气体$A_1$和$n_0\nu_2$mol的气体A$_2$的混合物在温度$T$和压强$p$下所占体积为$V_0$,当发生化学变化
\[
\nu_3A_3+\nu_4A_4-\nu_1A_1-\nu_2A_2=0
\]
并在相同的温度和压强下达到平衡时,其体积为$\tilde{V}$。试证明反应度$\varepsilon$为
\[
\varepsilon=\frac{\tilde{V}-V_0}{V_0}\cdot\frac{\nu_1+\nu_2}{\nu_3+\nu_4-\nu_1-\nu_2}
\]
\end{problem}
\begin{sol}
反应前状态方程
\begin{equation}
\label{4.10.1}pV_0=n_0(\nu_1+\nu_2)RT
\end{equation}
设反应度为$\varepsilon$,则反应后各气体组分的物质量
\begin{gather}
n_1=n_0\nu_1(1-\varepsilon)\\
n_2=n_0\nu_2(1-\varepsilon)\\
n_3=n_0\nu_3\varepsilon\\
n_3=n_0\nu_4\varepsilon
\end{gather}
反应后状态方程
\begin{equation}
\label{4.10.2}p\tilde{V}=(n_1+n_2+n_3+n_4)RT=n_0(\nu_1(1-\varepsilon)+\nu_2(1-\varepsilon)+\nu_3\varepsilon+\nu_4\varepsilon)RT
\end{equation}
联立式(\ref{4.10.1})和(\ref{4.10.2}),得
\begin{equation}
\varepsilon=\frac{\tilde{V}-V_0}{V_0}\cdot\frac{\nu_1+\nu_2}{\nu_3+\nu_4-\nu_1-\nu_2}
\end{equation}
\end{sol}
\end{document}