% !TEX program = pdflatex
% Statistical Physics Homework_5
\documentclass[12pt,a4paper]{article}
\usepackage[margin=1in]{geometry} 
\usepackage{amsmath,amsthm,amssymb,amsfonts,enumitem,fancyhdr,color,comment,graphicx,environ}
\pagestyle{fancy}
\setlength{\headheight}{65pt}
\newenvironment{problem}[2][Problem]{\begin{trivlist}
\item[\hskip \labelsep {\bfseries #1}\hskip \labelsep {\bfseries #2.}]}{\end{trivlist}}
\newenvironment{sol}
    {\emph{Solution:}
    }
    {
    \qed
    }
\specialcomment{com}{ \color{blue} \textbf{Comment:} }{\color{black}} %for instructor comments while grading
\NewEnviron{probscore}{\marginpar{ \color{blue} \tiny Problem Score: \BODY \color{black} }}
\usepackage[UTF8]{ctex}
\usepackage{multirow}
\lhead{Name: 陈稼霖\\ StudentID: 45875852}
\rhead{PHYS1503 \\ Statistical Physics \\ Semester Fall 2019 \\ Assignment 5}
\begin{document}
\begin{problem}{6.2}
试证明,对于一个一维自由粒子,在长度$L$内,在$\varepsilon$到$\varepsilon+d\varepsilon$的能量范围内,量子态数为
\[
D(\varepsilon)d\varepsilon=\frac{2L}{h}\left(\frac{m}{2\varepsilon}\right)^{1/2}d\varepsilon
\]
\end{problem}
\begin{sol}
在以$x$和$p$为直角坐标的$\mu$空间中,每个相格(量子态占据的格子)大小为
\begin{equation}
\Delta x\Delta p_x=h
\end{equation}
因此,$\mu$空间中的量子态密度为
\begin{equation}
\frac{1}{h}
\end{equation}
长度$L$和$\varepsilon$到$\varepsilon+d\varepsilon$的能量范围对应的$\mu$空间中的面积(考虑到动量$p_x$可以有正负两个方向)为
\begin{equation}
dxdp_x=2Ld(2m\varepsilon)^{1/2}=L\left(\frac{2m}{\varepsilon}\right)^{1/2}d\varepsilon
\end{equation}
在这块面积中的量子态数为
\begin{equation}
D(\varepsilon)d\varepsilon=\frac{1}{h}\cdot dxdp_x=\frac{2L}{h}\left(\frac{m}{2\varepsilon}\right)^{1/2}d\varepsilon
\end{equation}
\end{sol}

\begin{problem}{6.4}
在极端相对论情形下,粒子的能量动量$\varepsilon=cp$。试求在体积$V$内,在$\varepsilon$到$\varepsilon+d\varepsilon$的能量范围内三维粒子的量子态数。
\end{problem}
\begin{sol}
在体积$V$内,动量大小在$p$到$p+dp$的范围内(动量方向为任意),自由粒子可能的状态数为
\begin{equation}
\frac{4\pi V}{h^3}p^2dp
\end{equation}
根据公式$\varepsilon=cp$,在体积$V$内,在$\varepsilon$到$\varepsilon+d\varepsilon$的能量范围内,自由粒子可能的状态数为
\begin{equation}
\frac{4\pi V}{h^3}\left(\frac{\varepsilon}{c}\right)^2d\left(\frac{\varepsilon}{c}\right)=\frac{4\pi V\varepsilon^2}{h^3c^3}d\varepsilon
\end{equation}
\end{sol}

\begin{problem}{6.5}
设系统含有两种粒子,其粒子数分别为$N$和$N'$。粒子间的相互作用很弱,可以看做是近独立的。假设粒子可以分辨,处在一个个体量子态的粒子数不受限制。试证明,在平衡状态下两种粒子的最概然分布分别为
\[
a_l=\omega_le^{-\alpha-\beta\varepsilon_l}
\]
和
\[
a_l'=\omega_l'e^{-\alpha'-\beta\varepsilon_l'}
\]
其中$\varepsilon_l$和$\varepsilon_l'$是两种粒子的能级,$\omega_l$和$\omega_l'$是能级的简并度。\\
讨论:如果把一种粒子看作是一个系统,系统由两个子系统组成。以上结果表明,互为热平衡的两个子系统具有相同的$\beta$。
\end{problem}
\begin{sol}
由于粒子可分辨,且处在一个个体量子态的粒子数不受限制,故为玻尔兹曼分布,两种粒子的微观状态数分别为
\begin{gather}
\Omega=\frac{N!}{\prod_la_l!}\prod_l\omega_l^{a_l}\\
\Omega'=\frac{N'!}{\prod_la_l!}\prod_l\omega_l'^{a_l'}
\end{gather}
系统总微观状态数为
\begin{equation}
\Omega_{\text{总}}=\Omega\Omega'
\end{equation}
对上式两边同取对数得
\begin{equation}
\ln\Omega_{\text{总}}=\ln\Omega+\ln\Omega'=\ln N!-\sum_l\ln a_l!+\sum_l\ln\omega_l^{a_l}+\ln N'!-\sum_l\ln a_l'!+\sum_l\ln\omega_l'^{a_l'}
\end{equation}
当$a_l\gg1$时,利用斯特林公式得
\begin{align}
\nonumber\ln\Omega_{\text{总}}\approx&N(\ln N-1)-\sum_la_l(\ln a_l-1)+\sum_la_l\ln\omega_l\\
\nonumber&+N'(\ln N'-1)-\sum_la_l'(\ln a_l'-1)+\sum_la_l'\ln\omega_l'\\
\approx&N\ln N-\sum_la_l\ln a_l+\sum_la_l\ln\omega_l+N'\ln N'-\sum_la_l'\ln a_l'+\sum_la_l'\ln\omega_l'
\end{align}
当$a_l$和$a_l'$变化$\delta a_l$和$\delta a_l'$,$\ln\Omega_{\text{总}}$变化$\delta\ln\Omega_{\text{总}}$,为使$\ln\Omega_{\text{总}}$,
\begin{equation}
\delta\ln\Omega_{\text{总}}=-\sum_la_l\frac{1}{a_l}\delta a_l-\sum_l\ln a_l\delta a_l+\sum_l\ln\omega_lda_l-\sum_la_l'\frac{1}{a_l'}\delta a_l'-\sum_l\ln a_l'\delta a_l'+\sum_l\ln\omega_l'da_l'
\end{equation}
由于存在约束
\begin{gather}
\sum_la_l=N\Longrightarrow\delta N=\sum_l\delta a_l=0\\
\sum_la_l'=N'\Longrightarrow\delta N'=\sum_l\delta a_l'=0\\
\sum_l\varepsilon_la_l+\sum_l\varepsilon_l'a_l'=E\Longrightarrow\delta E=\sum_l\varepsilon_l\delta a_l+\sum_l\varepsilon_l'\delta a_l'=0
\end{gather}
设拉格朗日乘子$\alpha,\alpha',\beta$
\begin{gather}
\begin{align}
\nonumber\delta\ln\Omega_{\text{总}}-\alpha\delta N-\alpha'N'-\beta\delta E=&-\sum_l\left(\ln\frac{a_l}{\omega_l}+\alpha+\beta\varepsilon_l\right)\delta a_l\\
\nonumber&-\sum_l\left(\ln\frac{a_l'}{\omega_l'}+\alpha'+\beta\varepsilon'\right)\delta a_l'\\
=&0
\end{align}\\
\Longrightarrow\left\{\begin{array}{l}
\ln\frac{a_l}{\omega_l}+\alpha+\beta\varepsilon_l=0\\
\ln\frac{a_l'}{\omega_l'}+\alpha'+\beta\varepsilon_l'=0\\
\end{array}\right.\\
\Longrightarrow\left\{\begin{array}{l}
a_l=\omega_le^{-\alpha-\beta\varepsilon_l}\\
a_l'=\omega_l'e^{-\alpha'-\beta\varepsilon_l'}\\
\end{array}\right.
\end{gather}
如果把一种粒子看作是一个系统,互为热平衡的两个子系统具有相同的$\beta$,这是因为在热平衡的过程中两个子系统之间可以发生能量的交换。
\end{sol}

\begin{problem}{6.6}
同上题,如果粒子是玻色子或者费米子,结果如何?
\end{problem}
\begin{sol}
分类讨论:
\begin{itemize}
\item 设波色子粒子数为$N$,费米子粒子数为$N'$,则两种粒子的微观状态数分别为
\begin{gather}
\Omega=\prod_l\frac{(\omega_l+a_l-1)!}{a_l!(\omega_l-1)!}\\
\Omega'=\prod_l\frac{\omega_l'}{a_l'!(\omega_l'-a_l')!}
\end{gather}
对系统的微观状态数
\begin{equation}
\Omega_{\text{总}}=\Omega\Omega'
\end{equation}
取对数并利用斯特林公式得
\begin{equation}
\ln\Omega_{\text{总}}=\sum_l[(\omega_l+a_l)\ln(\omega_l+a_l)-a_l\ln a_l-\omega_l\ln\omega_l]+\sum_l[\omega_l'\ln\omega_l'-a_l'\ln a_l'-(\omega_l'-a_l')\ln(\omega_l'-a_l')]
\end{equation}
要使$\ln\Omega_{\text{总}}$极大,
\begin{equation}
\ln\Omega_{\text{总}}\approx\sum_l\ln\frac{(\omega_l+a_l)}{a_l}\delta a_l+\sum_l\ln\frac{(\omega_l'-a_l')}{a_l'}\delta a_l'=0
\end{equation}
根据约束
\begin{gather}
\sum_la_l=N\Longrightarrow\delta N=\sum_l\delta a_l=0\\
\sum_la_l'=N'\Longrightarrow\delta N'=\sum_l\delta a_l'=0\\
\sum_l\varepsilon_la_l+\sum_l\varepsilon_l'a_l'=E\Longrightarrow\delta E=\sum_l\varepsilon_l\delta a_l+\sum_l\varepsilon_l'\delta a_l'=0
\end{gather}
设拉格朗日乘子$\alpha,\alpha',\beta$
\begin{gather}
\begin{align}
\nonumber\delta\ln\Omega_{\text{总}}-\alpha\delta N-\alpha'\delta N'-\beta\delta E=&\sum_l\left(\ln\frac{(\omega_l+a_l)}{a_l}+\alpha+\beta\varepsilon_l\right)\delta a_l\\
\nonumber&+\sum_l\left(\ln\frac{(\omega_l'-a_l')}{a_l'}+\alpha'+\beta\varepsilon_l'\right)\delta a_l'\\
=&0
\end{align}\\
\Longrightarrow\left\{\begin{array}{l}
a_l=\frac{\omega_l}{e^{-\alpha-\beta\varepsilon_l}-1}\\
a_l'=\frac{\omega_l'}{e^{-\alpha'-\beta\varepsilon_l'}+1}\\
\end{array}\right.
\end{gather}
\item 若两种粒子均为波色子,设其粒子数分别为$N$和$N'$,则两种粒子的微观状态数分别为
\begin{gather}
\Omega=\prod_l\frac{(\omega_l+a_l-1)!}{a_l!(\omega_l-1)!}\\
\Omega'=\prod_l\frac{(\omega_l'+a_l'-1)}{a_l'!(\omega_l'-1)!}
\end{gather}
对系统的微观状态数
\begin{equation}
\Omega_{\text{总}}=\Omega\Omega'
\end{equation}
取对数并利用斯特林公式得
\begin{equation}
\ln\Omega_{\text{总}}=\sum_l[(\omega_l+a_l)\ln(\omega_l+a_l)-a_l\ln a_l-\omega_l\ln\omega_l]+\sum_l[(\omega_l'+a_l')\ln(\omega_l'+a_l')-a_l'\ln a_l'-\omega_l'\ln\omega_l']
\end{equation}
要使$\ln\Omega_{\text{总}}$极大,
\begin{equation}
\ln\Omega_{\text{总}}\approx\sum_l\ln\frac{(\omega_l+a_l)}{a_l}\delta a_l+\sum_l\ln\frac{(\omega_l'+a_l')}{a_l'}\delta a_l'=0
\end{equation}
根据约束
\begin{gather}
\sum_la_l=N\Longrightarrow\delta N=\sum_l\delta a_l=0\\
\sum_la_l'=N'\Longrightarrow\delta N'=\sum_l\delta a_l'=0\\
\sum_l\varepsilon_la_l+\sum_l\varepsilon_l'a_l'=E\Longrightarrow\delta E=\sum_l\varepsilon_l\delta a_l+\sum_l\varepsilon_l'\delta a_l'=0
\end{gather}
设拉格朗日乘子$\alpha,\alpha',\beta$
\begin{gather}
\begin{align}
\nonumber\delta\ln\Omega_{\text{总}}-\alpha\delta N-\alpha'\delta N'-\beta\delta E=&\sum_l\left(\ln\frac{(\omega_l+a_l)}{a_l}+\alpha+\beta\varepsilon_l\right)\delta a_l\\
\nonumber&+\sum_l\left(\ln\frac{(\omega_l'+a_l')}{a_l'}+\alpha+\beta\varepsilon_l'\right)\delta a_l'\\
=&0
\end{align}\\
\Longrightarrow\left\{\begin{array}{l}
a_l=\frac{\omega_l}{e^{-\alpha-\beta\varepsilon_l}-1}\\
a_l'=\frac{\omega_l'}{e^{-\alpha'-\beta\varepsilon_l'}-1}\\
\end{array}\right.
\end{gather}
\item 若两种粒子均为费米子,设其粒子数分别为$N$和$N'$,则两种粒子的微观状态数分别为
\begin{gather}
\Omega=\prod_l\frac{\omega_l}{a_l!(\omega_l-a_l)!}\\
\Omega'=\prod_l\frac{\omega_l'}{a_l'!(\omega_l'-a_l')!}
\end{gather}
对系统的微观状态数
\begin{equation}
\Omega_{\text{总}}=\Omega\Omega'
\end{equation}
取对数并利用斯特林公式得
\begin{equation}
\ln\Omega_{\text{总}}=\sum_l[\omega_l\ln\omega_l-a_l\ln a_l-(\omega_l-a_l)\ln(\omega_l-a_l)]+\sum_l[\omega_l'\ln\omega_l'-a_l'\ln a_l'-(\omega_l'-a_l')\ln(\omega_l'-a_l')]
\end{equation}
要使$\ln\Omega_{\text{总}}$极大,
\begin{equation}
\ln\Omega_{\text{总}}\approx\sum_l\ln\frac{(\omega_l-a_l)}{a_l}\delta a_l+\sum_l\ln\frac{(\omega_l'-a_l')}{a_l'}\delta a_l'=0
\end{equation}
根据约束
\begin{gather}
\sum_la_l=N\Longrightarrow\delta N=\sum_l\delta a_l=0\\
\sum_la_l'=N'\Longrightarrow\delta N'=\sum_l\delta a_l'=0\\
\sum_l\varepsilon_la_l+\sum_l\varepsilon_l'a_l'=E\Longrightarrow\delta E=\sum_l\varepsilon_l\delta a_l+\sum_l\varepsilon_l'\delta a_l'=0
\end{gather}
设拉格朗日乘子$\alpha,\alpha',\beta$
\begin{gather}
\begin{align}
\nonumber\delta\ln\Omega_{\text{总}}-\alpha\delta N-\alpha'\delta N'-\beta\delta E=&\sum_l\left(\ln\frac{(\omega_l-a_l)}{a_l}+\alpha+\beta\varepsilon_l\right)\delta a_l\\
\nonumber&+\sum_l\left(\ln\frac{(\omega_l'-a_l')}{a_l'}+\alpha'+\beta\varepsilon_l'\right)\delta a_l'\\
=&0
\end{align}\\
\Longrightarrow\left\{\begin{array}{l}
a_l=\frac{\omega_l}{e^{-\alpha-\beta\varepsilon_l}+1}\\
a_l'=\frac{\omega_l'}{e^{-\alpha'-\beta\varepsilon_l'}+1}\\
\end{array}\right.
\end{gather}
\end{itemize}
\end{sol}

\begin{problem}{*}
假定一种满足波色统计的分子可能占据的能级有$4$个,对应能级能量分别为$0$,$E$,$2E$,$3E$,能级简并度分别为$1$,$2$,$2$,$2$,如果系统含有$6$个分子,
\begin{itemize}
\item[a.] 写出与总能量$3E$相关的分布以及分布需要满足的条件,
\item[b.] 计算a)中每种分布对应的微观状态数,
\item[c.] 确定a)中每种分布的概率。
\end{itemize}
\end{problem}
\begin{sol}
\begin{itemize}
\item[a.] 与总能量相关的分布见表\ref{3E1}
\begin{table}[h]
\centering
\caption{与总能量$3E$相关的各种分布}
\label{3E1}
\begin{tabular}{|c|c|c|c|c|c|}
\hline
\multicolumn{2}{|c|}{分布}      & $a_0$ & $a_1$ & $a_2$ & $a_3$ \\ \hline
\multicolumn{2}{|c|}{能级}      & $0$   & $E$   & $2E$  & $3E$  \\ \hline
\multirow{3}{*}{可能情况序号} & $1$ & $3$   & $3$   & $0$   & $0$   \\ \cline{2-6} 
                        & $2$ & $4$   & $1$   & $1$   & $0$   \\ \cline{2-6} 
                        & $3$ & $5$   & $0$   & $0$   & $1$   \\ \hline
\end{tabular}
\end{table}
分布$\{a_n\}$所需要满足的条件
\begin{gather}
\sum_{l=0}^{3}a_l=a_0+a_1+a_2+a_3=6\\
\sum_{l=0}^{3}\varepsilon_la_l=Ea_1+2Ea_2+3Ea_3=E(a_1+2a_2+3a_3)=3E
\end{gather}
\item[b.] 分布$\{a_0,a_1,a_2,a_3\}=\{3,3,0,0\}$对应的微观状态数为
\begin{equation}
\Omega=\prod_l\frac{(\omega_l+a_l-1)!}{(\omega_l-a_l)!a_l!}=\frac{(1+3-1)!}{(1-1)!3!}\frac{(2+3-1)!}{(2-1)!3!}\frac{(2+0-1)!}{(2-1)!0!}\frac{(2+0-1)!}{(2-1)!0!}=4
\end{equation}
分布$\{a_0,a_1,a_2,a_3\}=\{4,1,1,0\}$对应的微观状态数为
\begin{equation}
\Omega=\prod_l\frac{(\omega_l+a_l-1)!}{(\omega_l-a_l)!a_l!}=\frac{(1+4-1)!}{(1-1)!4!}\frac{(2+1-1)!}{(2-1)!1!}\frac{(2+1-1)!}{(2-1)!1!}\frac{(2+0-1)!}{(2-1)!0!}=4
\end{equation}
分布$\{a_0,a_1,a_2,a_3\}=\{5,0,0,1\}$对应的微观状态数为
\begin{equation}
\Omega=\prod_l\frac{(\omega_l+a_l-1)!}{(\omega_l-a_l)!a_l!}=\frac{(1+5-1)!}{(1-1)!5!}\frac{(2+0-1)!}{(2-1)!0!}\frac{(2+0-1)!}{(2-1)!0!}\frac{(2+1-1)!}{(2-1)!1!}=2
\end{equation}
通过枚举法也可得到相同的结论,见表\ref{3E2}。
\begin{table}[h]
\centering
\caption{与总能量3E相关的各种微观状态下的各量子态分布的粒子数}
\label{3E2}
\begin{tabular}{|c|c|c|c|c|c|c|c|c|c|c|}
\hline
\multicolumn{4}{|c|}{能级}                                                                 & $0$ & \multicolumn{2}{c|}{$E$} & \multicolumn{2}{c|}{$2E$} & \multicolumn{2}{c|}{$3E$} \\ \hline
\multicolumn{4}{|c|}{简并态序号}                                                              & $1$ & $1$         & $2$        & $1$         & $2$         & $1$         & $2$         \\ \hline
\multirow{10}{*}{分布可能情况序号} & \multirow{4}{*}{$1$} & \multirow{10}{*}{微观状态数可能情况序号} & $1$  & $3$ & $3$         & $0$        & $0$         & $0$         & $0$         & $0$         \\ \cline{4-11} 
                           &                      &                               & $2$  & $3$ & $2$         & $1$        & $0$         & $0$         & $0$         & $0$         \\ \cline{4-11} 
                           &                      &                               & $3$  & $3$ & $1$         & $2$        & $0$         & $0$         & $0$         & $0$         \\ \cline{4-11} 
                           &                      &                               & $4$  & $3$ & $0$         & $3$        & $0$         & $0$         & $0$         & $0$         \\ \cline{2-2} \cline{4-11} 
                           & \multirow{4}{*}{$2$} &                               & $5$  & $4$ & $1$         & $0$        & $1$         & $0$         & $0$         & $0$         \\ \cline{4-11} 
                           &                      &                               & $6$  & $4$ & $1$         & $0$        & $0$         & $1$         & $0$         & $0$         \\ \cline{4-11} 
                           &                      &                               & $7$  & $4$ & $0$         & $1$        & $1$         & $0$         & $0$         & $0$         \\ \cline{4-11} 
                           &                      &                               & $8$  & $4$ & $0$         & $1$        & $0$         & $1$         & $0$         & $0$         \\ \cline{2-2} \cline{4-11} 
                           & \multirow{2}{*}{$3$} &                               & $9$  & $5$ & $0$         & $0$        & $0$         & $0$         & $1$         & $0$         \\ \cline{4-11} 
                           &                      &                               & $10$ & $5$ & $0$         & $0$        & $0$         & $0$         & $0$         & $1$         \\ \hline
\end{tabular}
\end{table}
\item[c.] 根据等概率原理,分布$\{a_0,a_1,a_2,a_3\}=\{3,3,0,0\}$的概率为
\begin{equation}
P_1=\frac{4}{4+4+2}=\frac{2}{5}
\end{equation}
分布$\{a_0,a_1,a_2,a_3\}=\{4,1,1,0\}$的概率为
\begin{equation}
P_1=\frac{4}{4+4+2}=\frac{2}{5}
\end{equation}
分布$\{a_0,a_1,a_2,a_3\}=\{5,0,0,1\}$的概率为
\begin{equation}
P_1=\frac{2}{4+4+2}=\frac{1}{5}
\end{equation}
\end{itemize}
\end{sol}
\end{document}